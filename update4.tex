\documentclass{article}
\usepackage[utf8]{inputenc}

\title{Distributed and Parallel Database Systems\\Update 4}
\author{Anikesh Jhadi \\19111007 \\VI semester BME \\NIT Raipur}

\date{\today}

\begin{document}

\maketitle

\section{Architectural Issues}
There are many possible distribution alternatives. The currently popular client/server architecture where a number of client machines access a single database server, is the most straightforward one. In these systems, which can be called multiple-client/single-server, the database management
problems are considerably simplified since the database is stored on a single server. 

The pertinent issues relate to the management of client buffers and the caching of data and (possibly) locks. The data management is done centrally at the single server.
A more distributed and more flexible architecture is the multiple-client/multiple server architecture
where the database is distributed across multiple servers which have to communicate with each other in
responding to user queries and in executing transactions. Each client machine has a “home” server to
which it directs user requests. The communication of the servers among themselves is transparent to the
users. Most current database management systems implement one or the other type of the client-server
architectures.


A truly distributed DBMS does not distinguish between client and server machines. Ideally, each site
can perform the functionality of a client and a server. Such architectures, called peer-to-peer, require
sophisticated protocols to manage the data distributed across multiple sites. The complexity of required
software has delayed the offering of peer-to-peer distributed DBMS products.


Parallel system architectures range between two extremes, the shared-nothing and the shared-memory
architectures. A useful intermediate point is the shared-disk architecture.


In the shared-nothing approach, each processor has exclusive access to itsmain memory and disk unit(s).
Thus, each node can be viewed as a local site (with its own database and software) in a distributed database
system. The difference between shared-nothing parallel DBMSs and distributed DBMSs is basically one
of implementation platform, therefore most solutions designed for distributed databases may be re-used in
parallel DBMSs. In addition, shared-nothing architecture has three main virtues: cost, extensibility, and
availability. On the other hand, it suffers from higher complexity and (potential) load balancing problems.


In the shared-memory approach, any processor has access to any memory module or disk unit through
a fast interconnect (e.g., a high-speed bus or a cross-bar switch). Several new mainframe designs such
as the IBM3090 or Bull’s DPS8, and symmetric multiprocessors such as Sequent and Encore, follow this
approach. Shared-memory has two strong advantages: simplicity and load balancing. These are offset by
three problems: cost, limited extensibility, and low availability.


In the shared-disk approach, any processor has access to any disk unit through the interconnect, but
exclusive (non-shared) access to its main memory. Each processor can then access database pages on the
shared disk and copy them into its own cache. To avoid conflicting accesses to the same pages, global
locking and protocols for the maintenance of cache coherency are needed. Shared-disk has a number of
advantages: cost, extensibility, load balancing, availability, and easy migration from uniprocessor systems.
On the other hand, it suffers from higher complexity and potential performance problems.





\end{document}
