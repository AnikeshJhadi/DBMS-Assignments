\documentclass{article}
\usepackage[utf8]{inputenc}

\title{Distributed and Parallel Database Systems\\Update 2}
\author{Anikesh Jhadi \\19111007 \\VI semester BME \\NIT Raipur}

\date{\today}

\begin{document}

\maketitle

\section{Underlying Principles}
A distributed database (DDB) is a collection of multiple, logically interrelated databases distributed over
a computer network. A distributed database management system is then defined as the software system that permits the management of the distributed database and makes the distribution
transparent to the users.These definitions point to two identifying architectural principles. The first is that the system consists of a (possibly empty) set of query sites and a non-empty set of data sites. The data sites have data storage capability while the query sites do not. The latter only
run the user interface routines in order to facilitate the data access at data sites. The second is that each
site (query or data) is assumed to logically consist of a single, independent computer. Therefore, each
site has its own primary and secondary storage, runs its own operating system (which may be the same
or different at different sites), and has the capability to execute applications on its own. The sites are
interconnected by a computer network rather than a multiprocessor configuration. The important point here
is the emphasis on loose interconnection between processors which have their own operating systems and
operate independently.

The database is physically distributed across the data sites by fragmenting and replicating the data. Given a relational database schema, fragmentation subdivides each relation into horizontal or vertical partitions. Horizontal fragmentation of a relation is accomplished by a selection operation
which places each tuple of the relation in a different partition based on a fragmentation predicate (e.g., an
Employee relation may be fragmented according to the location of the employees). Vertical fragmentation,
divides a relation into a number of fragments by projecting over its attributes (e.g., the Employee relation
may be fragmented such that the Emp number, Emp name and Address information is in one fragment,
and Emp number, Salary and Manager information isin another fragment). Fragmentation is desirable
because it enables the placement of data in close proximity to its place of use, thus potentially reducing
transmission cost, and it reduces the size of relations that are involved in user queries.

Based on the user access patterns, each of the fragments may also be replicated. This is preferable when
the same data are accessed from applications that run at a number of sites. In this case, it may be more
cost-effective to duplicate the data at a number of sites rather than continuously moving it between them.

When the above architectural assumptions of a distributed DBMS are relaxed, one gets a parallel
database system. The differences between a parallel DBMS and a distributed DBMS are somewhat unclear.
In particular, shared-nothing parallel DBMS architectures, which we discuss below, are quite similar to
the loosely interconnected distributed systems. Parallel DBMSs exploit recent multiprocessor computer
architectures in order to build high-performance and high-availability database servers at a much lower price
than equivalent mainframe computers.



\end{document}
