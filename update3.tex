\documentclass{article}
\usepackage[utf8]{inputenc}

\title{Distributed and Parallel Database Systems\\Update 3}
\author{Anikesh Jhadi \\19111007 \\VI semester BME \\NIT Raipur}

\date{\today}

\begin{document}

\maketitle


There are a number of identifying characteristics of the distributed and parallel DBMS technology.

1. The distributed/parallel database is a database, not some “collection” of files that can be individually
stored at each node of a computer network. This is the distinction between a DDB and a collection
of files managed by a distributed file system. To form a DDB, distributed data should be logically
related, where the relationship is defined according to some structural formalism (e.g., the relational
model), and access to data should be at a high level via a common interface.

2. The system has the full functionality of a DBMS. It is neither, as indicated above, a distributed file
system, nor is it a transaction processing system. Transaction processing is only one of the functions
provided by such a system, which also provides functions such as query processing, structured
organization of data, and others that transaction processing systems do not necessarily deal with.

3. The distribution (including fragmentation and replication) of data across multiple site/processors is
not visible to the users. This is called transparency. The distributed/parallel database technology
extends the concept of data independence, which is a central notion of database management, to
environments where data are distributed and replicated over a number of machines connected by a
network. This is provided by several forms of transparency: network (and, therefore, distribution)
transparency, replication transparency, and fragmentation transparency. Transparent access means
that users are provided with a single logical image of the database even though it may be physically
distributed, enabling them to access the distributed database as if it were a centralized one. In its
ideal form, full transparency would imply a query language interface to the distributed/parallel DBMS
which is no different from that of a centralized DBMS.




\end{document}
