\documentclass{article}
\usepackage[utf8]{inputenc}

\title{Distributed and Parallel Database Systems}
\author{Anikesh Jhadi \\19111007 \\VI semester BME \\NIT Raipur}

\date{\today}

\begin{document}

\maketitle

\section{Introduction}
The maturation of database management system (DBMS) technology has coincided with significant developments in distributed computing and parallel processing technologies. The end result is the emergence of distributed database management systems and parallel database management systems. These systems
have started to become the dominant data management tools for highly data-intensive applications.

The integration of workstations in a distributed environment enables a more efficient function distribution in which application programs run on workstations, called application servers, while database functions are handled by dedicated computers, called database servers. This has led to the present trend in distributed system architecture, where sites are organized as specialized servers rather than as general-purpose
computers.

A parallel computer, or multiprocessor, is itself a distributed system made of a number of nodes connected by a fast network within a cabinet. Distributed database technology can
be naturally revised and extended to implement parallel database systems, i.e., database systems on parallel
computers. Parallel database systems exploit the parallelism in
data management in order to deliver high-performance and high-availability database servers at a much lower price than equivalent mainframe computers.



\end{document}
